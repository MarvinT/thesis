Parametrizing complex natural stimuli is a difficult and long-standing challenge. We used a generative deep convergent network to represent and parametrize a large corpus of song from European starlings, a songbird species, into a compressed low-dimensional space. We applied psychophysical methods to probe categorical perception of natural starling song syllables, which reveal a shared categorical perceptual space. Some categorical boundaries are sensitive to the category assignment of training syllables, indicating that the consensus is context dependent and that underlying dimensions of the space are not independent. We record simultaneous firing from populations of 10's of neurons in a secondary auditory cortical region of anesthetized starlings. By estimating how fast population level neural representation change with respect to the stimuli, we produce a measure along a path in stimuli space that is shared between birds and descriptive of the psychophysically determined parameters in other birds. Consistent with this, we predict the behavioral psychometric function along one dimension by fitting the behavior for other dimensions to the population level neural activity. Thus, knowing how the animal responds in one sub-region of the parametrized space informs responses in other sub-regions. Our results implicate the importance of experience in shaping shared perceptual boundaries among complex communication signals and suggest the categorical representation of natural signals in secondary sensory cortices is distributed much more densely than predicted by traditional hierarchical object recognition models.