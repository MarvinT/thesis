\acf{CP} is the phenomenon where categories held by the observer, either learned or inherent, influence the observer's perception. This means that the perception of changes in the stimuli does not depend solely on the physical changes in the stimuli. Small changes in stimuli that move near or across category boundaries are very noticeable while more substantial changes in other regions may not be noticeable at all. This means that our perceptual system transforms relatively linear sensory signals into relatively nonlinear interval representations.

Experimentally we observe \CP as demonstrated in figure (\ref{fig:outline}A). For a defined physical continuum that moves between two (or more) categories, A and B, for example, we can experimentally ask a participant to classify the stimuli as belonging to category A or category B. This provides the sigmoid-shaped psychometric curve shown in green in figure (\ref{fig:outline}A) where the probability of classifying a stimulus as category B is high when the stimulus is near the B category and low when the stimulus is near the A category. Another \CP hallmark is a peak in the discrimination function near the boundary of two categories. This is tested by experimentally asking a participant to distinguish nearby stimuli on the physical continuum, and is shown in blue in figure (\ref{fig:outline}A). Thus, the physical continuum is perceived as being stretched near the category boundary and compressed within category limits.

The conventional view of the auditory object processing pathway (Ventral Auditory Pathway) is very similar to the visual object processing pathway (Ventral Visual Pathway). The ventral auditory pathway consists of the hierarchical processing of categorical information where neurons become increasingly sensitive to more complex stimuli and abstract information between the beginning stages and the latter stages. It is therefore theorized that as you move along the ventral auditory pathway, there is a progression of category information processing, where simpler receptive fields are combined to form more and more complex receptive fields encoding more and more complex features of the stimuli until you reach the auditory equivalent of a ``grandmother cell.'' Perhaps not quite as extreme as proposing the existence of the auditory equivalent of a ``grandmother cell,'' there are regions such as the superior temporal gyrus and sulcus which is categorically activated by speech sounds, relative to other sounds \cite{Binder2000, leaver2010cortical} and there are distinct regions of superior temporal gyrus and sulcus that are selectively activated by musical instruments sounds \cite{leaver2010cortical} and even tool sounds and birdsong \cite{doehrmann2008probing}.

European starlings ({\it Sturnus vulgaris}) are an excellent established model organism to study auditory processing and categorical perception. Like human speech, starling song is composed of learned, spectrally complex, temporally-patterned acoustic objects (called \textit{\textbf{motifs}}), that are produced in long, well-organized temporal sequences \cite{gentner2003neuronal}, and that function in a wide range of natural behaviors. As with other complex natural signals, our understanding of how birdsongs are represented in higher cortical regions, both benefits from and is hindered by the complex spectro-temporal character of these sounds. Multiple physiological studies have used conspecific vocalizations, and reveal a strong selectivity for songs that emerges across the auditory forebrain and strengthens from field L to \ac{NCM} and \ac{CM} \cite{gentner2003neuronal, gentner2004neural, thompson2010song, jeanne2011emergence}.

Starlings rely on object-level representations of motifs \cite{Meliza2010,gentner1998perceptual,Comins2013,comins2014auditory}, and the neural correlates of motif perception are localized in the regions of the auditory forebrain \emph{that are evolutionarily homologous to the mammalian auditory cortex\cite{wang2010laminar}}. The songbird auditory system follows the vertebrate plan\cite{Carr1992}. Field L2a is the primary telencephalic target of the auditory thalamus\cite{Karten1968}, and is the input layer for a columnar circuit spanning L1, L3, and \ac{CM}, anatomically\cite{Wang2010a,Jarvis2005}, genetically\cite{Dugas-Ford2012}, and functionally\cite{Calabrese2015} analogous to the mammalian auditory cortical micro-circuit. Field L sub-regions also project to the \ac{NCM}, a region that, along with the \ac{CLM}, shares reciprocal connections with the \ac{CMM}. Neurons in \ac{NCM} have been shown to have complex composite receptive fields\cite{kozlov2016central} that are reminiscent of the multidimensional receptive fields found in cat A1\cite{atencio2008cooperative}. Multidimensional receptive field characteristics can be reproduced by a \ac{DNN}\cite{kozlov2016central} trained to represent starling songs. Neurons throughout the avian cortex are selective for species-specific (conspecific) vocalizations\cite{Bonke1979,Leppelsack1976,Muller1985}, with selectivity increasing from Field L2, to L1 and L3\cite{Theunissen2004,Theunissen1998,Theunissen2000}, and again in \ac{NCM} and \ac{CM}\cite{Calabrese2015,Muller1985,Grace2003,Bonke1979,Leppelsack1976,gentner2003neuronal,Gentner2004,Thompson2010,Jeanne2011}. This is consistent with a functional hierarchy tuned to conspecific song\cite{Hsu2004,Woolley2005}, refined by experience\cite{gentner2003neuronal,Sockman2002,Sockman2005,Phan2006,Thompson2010,Jeanne2011}, with selectivity for acoustic features at multiple timescales increasing in a feed-forward manner\cite{Rose1988,Kaas2000,Binder2000,VanEssen1992}. Because of its compact organization, close relation to well-studied sensorimotor/vocal production structures, and the rich behavioral history of birdsong, the system is ideal for understanding the processing of natural acoustic communication signals.
Starlings respond to natural complex stimuli patterns such as the presence or absence of a center-embedded recursive structure\cite{gentner2006recursive,comins2014auditory,comins2014temporal}, a characteristic previously thought unique to human language (at least by Chomsky), allowing access to signals that would be invasive in humans, and cost prohibitive in non-human primates.

\subsection{The difficulties of using European Starlings}

The lack of parametric control over the complex acoustic features composing birdsongs (and other communication signals in other species) had rendered it difficult to more rigorously and extensively characterize both the information that these regions encode, and how this information is encoded. Ideally, we would like to parametrically control the complex natural stimuli to which high-order sensory regions are tuned, with the same precision and control that past studies have manipulated more simple stimuli like white noise and simple sine wave stimuli that can drive more primary sensory regions.

Furthermore, starlings are not among the most popular model organisms, so many of the off the shelf tools available for other model organisms haven't been established for the Starlings. This includes the genetic and viral tools as well as off the shelf hardware and tools.

Lastly, increased mobility (3-dimensional freedom and frequent backflips) makes recordings while behaving more difficult, even though there have been examples of it\cite{knudsen2013active,bluvas2013attention}.

\subsection{What is required for \CP research?}

The fundamental difficulty with \CP research is that it generally occurs in the perception of a high dimensional space where naturally occurring ecologically relevant stimuli exist only on non-uniformly distributed sub-manifold of that space. In terms of auditory \CP in human speech, the high dimensional space is the space of all possible sounds that could possibly be heard, most of which would sound like random noise. The naturally occurring ecologically relevant stimuli are all the possible spoken sounds or phonemes that are useful for human speech production and comprehension, which exist in a much smaller lower-dimensional subset of all possible sounds. Even within this subset of sounds, not all possible phonemes are heard with equal probability, and this distributional information may begin to shape our perceptual systems even before birth.

Because of this, the majority of auditory \CP has focused on human speech sounds, predominantly English, leveraging our intuitions and years of linguistic modeling to identify the relevant dimensions where \CP occur. \CP work concerning animal vocalizations must first identify stimuli that are categorically perceived, a process that was very labor intensive\cite{nelson1989categorical,prather2009neural,lachlan2015context} before the development of more modern techniques\cite{sainburg2019parallels}. The variation within the songs of European starlings also makes this task easier because clustering song elements is more straightforward with more distinct clusters.

Furthermore, the identification of category boundaries is more difficult in non-human research than with human subjects. Since we can't provide instructions, we are forced to use multiple shaping or pre-training stages of operant behavioral conditioning. Also, if we permit a fully adaptive double staircase procedure, our subjects learn to exploit this and force the boundary all the way to one side or another. To solve this, we use a new staircase technique.

Last, but certainly not least, \CP requires the generation of a synthetic continuum between the categories or category exemplars. In human speech, the existence of speech synthesizers has made this task easy, at least in English. In Chinese for example, the technical difficulty in creating a Chinese-based synthetic continuum suitable for a \CP study has been proposed as a reason for the large number of papers focused on \CP in lexical tones in the Chinese language\cite{zhang2013categorical}. \CP research on animal vocalization has been limited to exploiting natural variation in recorded calls\cite{nelson1989categorical,prather2009neural,lachlan2015context} which invariably limits the sampling and control over the stimuli. It has also been shown that the quality of the speech synthesizer is correlated with how much \CP is observed\cite{van1999categorical}.

Thus the approach we present streamlines the following steps for \CP research in non-human vocalizations:
\begin{enumerate}
    \item Identification of relevant dimensions where \CP occurs
    \item Identification of categorical boundaries
    \item Generation of a synthetic continuum between the categories/ category exemplars.
\end{enumerate}