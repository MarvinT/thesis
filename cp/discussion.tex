Our results overcome a long-standing impediment to understanding the perception of natural communication signals. We demonstrate a method for parametrizing complex stimuli and generating \emph{smoothly varying morphs between these stimuli}, as well as how to use these morphs to explore the perceptual basis, behaviorally and neurally, of the natural stimulus space. To our knowledge, this marks one of the first naturalistic parametric explorations of non-human auditory communication signals. Our characterization of the perception of this space and its neurological underpinnings, reveals remarkable behavioral consensus between animals for categorical boundaries and a broadly distributed encoding strategy for categorical stimulus information at the neural population level.  

The observed shift in psychometric boundaries between cohorts was difficult to statistically explore because they only occurred in three of the 24 of the morph dimensions measured between the three cohorts. These shifts may indicate that depending on initial training conditions, birds may end up using different stimuli features to perform the classification or that discrimination along certain features may be altered by learning.

The work demonstrates the existence of a shared perceptual space, common across individuals, in which perceived categorical boundaries cluster at consensus locations.  This kind of consensus is a pre-requisite to functional communication systems that use discrete signals.  Furthermore, the 16 interpolating morph dimensions used in this study are not independent, nor are they a simple linear function of the endpoints, independent of the structure of the network space. If the latter were true, then all (or none) of the boundaries would shift when the endpoints were permuted. Instead, because only some of the dimensions are affected by permutation of the initial categories, not all the dimensions are independent, and the relationships between them are likely complex. Moreover, because there are dimensions that are not changed by the permutation of the initial categories, the decision boundaries learned by birds cannot rely on simple separation of the initial template motifs (as are seen in algorithms such as support vector machines or equivalent). Understanding where these boundaries fall likely requires knowledge of how natural stimuli is distributed in the latent space of the network, and the underlying geometry in which the latent manifold is embedded. Additional work is needed in these areas. \cite{tims paper}

The field of machine learning is rapidly evolving and there are number of possible improvements to the processing and methodology, however, this work mainly demonstrates the usefulness of these kinds of techniques for understanding the perception of complex natural communication signals. In addition to changes in network architecture, newer implementations of spectrogram inversion would improve stimulus generation and are currently being tested and developed. In our experiment, however, different initializations of the spectrogram inversion process revealed that neurons in \ac{CM} are sensitive to these small physical differences in physical stimuli that are imperceptible to human ears figure \ref{fig:tsne}.

The network representation decoding of the psychophysical parameters indicates the importance of the distribution of natural songs on the perceptual space. While our library of songs provides for an approximation of their natural experience history, it is limited, especially in the context of wild caught animals. A fascinating future direction of this work would be to see if changing the distribution of experienced song before the training would influence category boundaries.

Our ability to predict held out behavioral responses from both the artificial neural network activations as well as the in vivo neural population activities indicate that each of these representations are sufficient, although not uniquely so, to describe the perceptual and behavioral space. One compelling result is that the behavior generalizes across this stimulus space, such that that knowing how perception acts in one sub-region can inform behavioral responses in other sub-regions. This category generalization also deserves further study.

Finally, the fact that categorical behavioral responses can be decoded from a randomly selected set of 10s of neurons contributes to a growing body of work \cite{jeanne2011emergence,kozlov2016central} that opposes the strongest version of sparse hierarchical models of perception, where neurons with simpler receptive fields converge onto neurons with a more complex receptive fields until a complex percept, like Jennifer Aniston emerges \cite{quiroga2005invariant}. Under this model, decoding a categorical behavioral measure (as we show here) is only possible by matching the right stimuli to the right subset of neurons. Thus, our results imply that \emph{the representation of these secondary auditory regions is much more distributed than would be predicted by a model where increasingly complex features are encoded exclusively by single neurons}.

Using A DBN as a generative model of birdsong is another limitation of this work due to the amount of recent progress in machine learning. In particular, we recommend using \cite{GAIA} for these techniques as it can produce much more realistic sounding birdsong and is explicitly constructed such that the stimuli space is convex and interpolations between exemplars are realistic.

Another possible limitation of these results is the anesthetized recording prep that was used. These measurements might not necessarily reflect the actual awake representations, and especially attentional modulation but certainly measure the functional tendencies of the network representation\cite{Emily's attention paper?, Dan's chronic paper?}. Additionally, this doesn't change our interpretation which is based on the fact that we can predict categorical boundary parameters and not precisely what representation we are using to make the prediction.

While regions of the auditory forebrain are homologous to mammalian auditory cortex\cite{wang2010laminar}, there are significant structural differences. Thus, while solutions to categorical perception of complex auditory stimuli used by birds aren't necessarily the same as those in our own, they do provide another example of a neural system capable of solving problems (recognizing complex hierarchical features and recursive grammars) previously thought to be strictly within human capacity. Thus, while it might not allow exploration of the neural circuitry we use, this line of work can perhaps indicate what kinds of circuit patterns and architectures could be useful in a system that needs this capacity.