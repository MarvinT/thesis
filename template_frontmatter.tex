%
%
% UCSD Doctoral Dissertation Template
% -----------------------------------
% http://ucsd-thesis.googlecode.com
%
%


%% REQUIRED FIELDS -- Replace with the values appropriate to you

% No symbols, formulas, superscripts, or Greek letters are allowed
% in your title.
\title{The Unreasonable Effectiveness of Machine Learning in Neuroscience: \\Understanding High-dimensional Neural Representations with Realistic Synthetic Stimuli}

\author{Marvin Thielk}
\degreeyear{\the\year}

% Master's Degree theses will NOT be formatted properly with this file.
\degreetitle{Doctor of Philosophy}

\field{Neurosciences}
\specialization{Computational Neuroscience}  % If you have a specialization, add it here

\chair{Professor Tim Gentner}
% Uncomment the next line iff you have a Co-Chair
% \cochair{Tatyana Sharpee}
%
% Or, uncomment the next line iff you have two equal Co-Chairs.
%\cochairs{Professor Chair Masterish}{Professor Chair Masterish}

%  The rest of the committee members  must be alphabetized by last name.
\othermembers{
Professor Saket Navlakha\\
Professor Terrence Sejnowski\\
Professor John Serences\\
Professor Tatyana Sharpee
}
\numberofmembers{5} % |chair| + |cochair| + |othermembers|


%% START THE FRONTMATTER
%
\begin{frontmatter}

%% TITLE PAGES
%
%  This command generates the title, copyright, and signature pages.
%
\makefrontmatter

%% DEDICATION
%
%  You have three choices here:
%    1. Use the ``dedication'' environment.
%       Put in the text you want, and everything will be formated for
%       you. You'll get a perfectly respectable dedication page.
%
%
%    2. Use the ``mydedication'' environment.  If you don't like the
%       formatting of option 1, use this environment and format things
%       however you wish.
%
%    3. If you don't want a dedication, it's not required.
%
%
\begin{dedication}
  To Alex, for reminding me not to sacrifice too much of my physical, mental, and emotional well-being as I tried to find out how deep the rabbit hole goes.
  \\And to my family, especially my parents, who taught me to focus on three things:
  \\Do good. Work hard. And never stop learning and improving.
  \newline \newline \newline
  ``Always take time to sharpen your axe.''
\end{dedication}


% \begin{mydedication} % You are responsible for formatting here.
%   \vspace{1in}
%   \begin{flushleft}
% 	To me.
%   \end{flushleft}
%
%   \vspace{2in}
%   \begin{center}
% 	And you.
%   \end{center}
%
%   \vspace{2in}
%   \begin{flushright}
% 	Which equals us.
%   \end{flushright}
% \end{mydedication}



%% EPIGRAPH
%
%  The same choices that applied to the dedication apply here.
%
\begin{epigraph} % The style file will position the text for you.
  \emph{If I have seen further it is by standing on the shoulders of giants.}\\
  ---Isaac Newton\\
  \vspace{1in}
  \emph{There is only one thing which is more unreasonable than the unreasonable effectiveness of mathematics in physics, and this is the unreasonable ineffectiveness of mathematics in biology.}\\
  ---Israel Gelfand
\end{epigraph}

% \begin{myepigraph} % You position the text yourself.
%   \vfil
%   \begin{center}
%     {\bf Think! It ain't illegal yet.}
%
% 	\emph{---George Clinton}
%   \end{center}
% \end{myepigraph}


%% SETUP THE TABLE OF CONTENTS
%
\tableofcontents
\listoffigures  % Comment if you don't have any figures
% \listoftables   % Comment if you don't have any tables



%% ACKNOWLEDGEMENTS
%
%  While technically optional, you probably have someone to thank.
%  Also, a paragraph acknowledging all coauthors and publishers (if
%  you have any) is required in the acknowledgements page and as the
%  last paragraph of text at the end of each respective chapter. See
%  the OGS Formatting Manual for more information.
%
\begin{acknowledgements}
 Thanks to the universe, without which, this probably wouldn't exist.
 
 Thanks to Alex, for telling me if what I'm generating is real or not.
 
 Thanks to my parents, Mike and Minh, for making me the man I am today.
 
 Thanks to my brothers, Aaron and Alex, for reminding me that the people who care don't matter and the people who matter don't care. RL is much better in co-op.
 
 Thanks to my PI, Tim, for years of both scientific and non-scientific guidance, and for sometimes knowing what I needed more than I did, like when he came into lab at 11:30 PM on Friday night to help me without me having to ask. I am the scientist I am today because of him.
 
 Thanks to my lab. Dan, Justin, and Krista, for being my role models and creating the environment that made me join the lab. Brad, for being the opposite of me in a lot of ways, but the same as me in all the best ways. Tim, for noticing I say wavenet at every presentation. Michael, for doing all the things I've never even tried. Zeke, for being the postdoc I never knew I needed. My undergrads, Karen, Kevin, Mingcheng, and Darvesh, for helping me and reminding me how much I enjoy mentorship. Nasim, Sasen, Kai, Srihita, Anna, Daril, and Sean, for being a constant source of encouragement and reminding me of all the progress I've made during grad school.
 
 Thanks to my cohort, especially Aki, Andrea, Cailey, Claire, Ethan, Geoff, Geoff, Kathleen, Kyle, Landon, Laura, Matt, Melissa, Sequoyah, Stephen, Tom, and Wilmer, for making UCSD NGP the best and getting me into the right amount of trouble and knowing when to ``just let him go.'' I'll never forget running from helicopter spotlight with you guys.
 
 Thanks to my past-life friends, Allic, Michael, Yuliy, Hernan, Will, Bryan, and Cam, for bursting my bubble and knowing I'm an introvert pretending to be an extrovert.
 
 Thanks to Erin and Linh, for solving all the problems I caused.
 
 Finally, thanks to my co-PI, Tanya, for providing theoretical spice to my work, as well as the rest of my committee, Terry, John, and Saket, for their guidance and providing reference to my work.
 
 \clearpage
 
 Chapter 1, in part, is currently being prepared for submission for publication of the material. Thielk, Marvin; Sainburg, Timothy; Sharpee, Tatyana; Gentner, Timothy. The dissertation author was the primary investigator and author of this manuscript
\end{acknowledgements}


%% VITA
%
%  A brief vita is required in a doctoral thesis. See the OGS
%  Formatting Manual for more information.
%
\begin{vitapage}
\begin{vita}
  \item[2012] B.~A. with majors in Applied Mathematics \emph{with honors} and Computer Science \emph{with honors}, University of California, Berkeley
  \item[2019] Ph.~D. in Neurosciences with a Specialization in Computational Neuroscience, University of California, San Diego
\end{vita}
% \begin{publications}
%   \item Your Name, ``A Simple Proof Of The Riemann Hypothesis'', \emph{Annals of Math}, 314, 2007.
%   \item Your Name, Euclid, ``There Are Lots Of Prime Numbers'', \emph{Journal of Primes}, 1, 300 B.C.
% \end{publications}
\end{vitapage}


%% ABSTRACT
%
%  Doctoral dissertation abstracts should not exceed 350 words.
%   The abstract may continue to a second page if necessary.
%
\begin{abstract}
  Parametrizing complex natural stimuli is a difficult and long-standing challenge. We used a generative deep convergent network to represent and parametrize a large corpus of song from European starlings, a songbird species, into a compressed low-dimensional space. We applied psychophysical methods to probe categorical perception of natural starling song syllables, which reveal a shared categorical perceptual space. Some categorical boundaries are sensitive to the category assignment of training syllables, indicating that the consensus is context dependent and that underlying dimensions of the space are not independent. We record simultaneous firing from populations of 10's of neurons in a secondary auditory cortical region of anesthetized starlings. By estimating how fast population level neural representation change with respect to the stimuli, we produce a measure along a path in stimuli space that is shared between birds and descriptive of the psychophysically determined parameters in other birds. Consistent with this, we predict the behavioral psychometric function along one dimension by fitting the behavior for other dimensions to the population level neural activity. Thus, knowing how the animal responds in one sub-region of the parametrized space informs responses in other sub-regions. Our results implicate the importance of experience in shaping shared perceptual boundaries among complex communication signals and suggest the categorical representation of natural signals in secondary sensory cortices is distributed much more densely than predicted by traditional hierarchical object recognition models. This thesis also explores other applications of machine learning to solve neuroscience problems, in particular, the curse of dimensionality and exploring predictive coding and surprise. A model explicitly designed to predict future states allows the compression of high-dimensional time-varying signals into a lower-dimensional representation encoding exclusively predictive and predictable information and has many practical applications.
\end{abstract}


\end{frontmatter}
